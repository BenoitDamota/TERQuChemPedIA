\subsection{Motivation}
\par La matrice réduite des distances à des points fixes a pour objectif de corriger les défauts de la représentation géométrique moléculaire par matrice réduite des distances inter-atomiques (REF MATR RED DIST REL). Cette dernière possédait en effet le défaut majeur de ne pas être systématiquement réversible en matrice des coordonnées atomiques (REF REPR MAT COORDS). Ce défaut était dû à la propagation des erreurs induite par le fait que les positions des atomes étaient calculées à partir du calcul de la position des atomes précédents (REF REPR DIST REL RECONSTRUCT). Pour parer cela, nous définissons une représentation telle que la position de chaque atome est définie à partir de distances à quatre points fixes du repère. Les erreurs, même si elles existent toujours à des valeurs minimes (autour de 10^{-25} m), ne se propagent donc plus lors de reconstruction des positions des atomes.\\
Un autre problème résolu par cette nouvelle représentation est qu'il n'existe plus de molécule dont on ne peut pas reconstruire les positions à cause d'une géométrie plane ou linéaire (REF AT FICTIF), le calcul de la position de chaque atome dépendant désormais de la distance à quatre points de l'espace que l'on choisit tels qu'ils n'appartiennent pas à un même plan.\\

\subsection{Formalisation}