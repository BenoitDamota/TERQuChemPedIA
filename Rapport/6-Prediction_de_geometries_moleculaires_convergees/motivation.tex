\par L'objectif des modèles prédictifs que l'on décrit dans ce chapitre est de prédire la géométrie convergée (REF GEOM CONVERG) d'une molécule complète, à partir d'une géométrie non convergée. Ils sont issus d'une tentative de reproduction de résultats antérieurs, afin de confirmer la méthode élaborée lors des stages précédents sur le projet QuChemPedIA.\\
Chronologiquement, ces modèles ont constitué la première partie de mon travail, avant de passer aux modèles tentant de prédire les longueurs de liaisons (REF DIST\_REL), à cause de l'impossibilité de produire des prédictions de qualité suffisante (REF RESULTATS).\\

\par L'objectif à terme de ces modèles est de pouvoir constituer une alternative au DFT (REF DFT) pour calculer rapidement la géométrie convergée d'une molécule. Cela nécessite de produire des prédictions d'une très grande précision. Cependant, le but ici est avant tout de valider une méthode et notre capacité à produire des prédictions d'ordre géométrique. Nous ne cherchons donc pas à créer un modèle effectuant de très bonnes prédictions, mais plutôt à définir une représentation des données et un ensemble de paramètres permettant d'obtenir de bons résultats.

\paragraph{Introduction de bruit} Afin de prédire des géométries moléculaires convergées à partir de géométries moléculaires non convergées, la situation idéale serait que les modèles apprennent à partir d'un ensemble de géométries non convergées issues de mesures ou d'optimisation par mécanique moléculaire (REF MÉCA MOL), et l'ensemble de géométries convergées par le DFT (REF DFT) associé. Cela constituerait en effet un ensemble de données homogène qui aurait l'avantage d'être comparable aux données que l'on utiliserait dans un cas d'utilisation réel.\\
Malheureusement, nous ne possédons pas de telles données. Nous possédons les géométries convergées issues de la base PubChemQC (REF PUBCHEMQC) mais pas les géométries à partir desquelles elles ont été calculées. S'il est théoriquement possible de calculer la géométrie optimisée en mécanique moléculaire de toutes les molécules de la base PubChemQC en utilisant le programme Open Babel\footnote{\url{http://openbabel.org/wiki/Main_Page}}, la perte de l'ordre des atomes lors de l'optimisation rend la procédure impossible en pratique.\\
L'alternative retenue lors des stages précédents est d'introduire du bruit (REF PREP DONNEES BRUIT) dans les coordonnées des géométries optimisées, et d'entraîner les modèles à prédire ce bruit. La différence entre la géométrie bruitée et le bruit prédit permet alors d'obtenir la géométrie optimisée par le modèle. L'introduction de bruit ne garantit donc pas que les modèles se généraliseront aux données réelles, mais semble tout de même raisonnable pour tenter de valider la méthode, puisque nous entraînons des modèles dont l'objectif est de déplacer les atomes d'une molécule de sorte à obtenir une géométrie convergée.

\paragraph{Modèles} Cinq modèles différents ont été entraînés. Ils diffèrent par les représentations utilisées en entrée et en sortie, les caractéristiques des molécules dont on tente de prédire la géométrie convergée, et les paramètres propres aux réseaux de neurones comme les fonctions de coût ou la topologie. Nous allons répertorier les différentes caractéristiques utilisées mais pas modèle par modèle, puisque aucun ensemble de caractéristiques n'a produit de résultats significativement meilleurs que les autres (REF RESULTATS). Cependant, une table des caractéristiques utilisées modèle par modèle est disponible en annexe (REF CARAC ANNEXES).


