Afin de simplifier leur dénomination, on nomme les différents modèles prédictifs. Tous les modèles décrits dans ce chapitre ont un nom de préfixe « DELTA\_DIST\_+H », issu de leur vocation à prédire des différences ($\Delta$) de distances pour obtenir des géométries convergées. Le suffixe « +H » indique que les données d'entrée contiennent les informations concernant les atomes d'hydrogène. Initialement, des modèles ne contenant pas les atomes d'hydrogène en entrée devaient être créés par la suite, mais ce projet a été abandonné faute de pouvoir obtenir des résultats satisfaisants avec le modèle courant (\ref{delta_dist_abandon}).\\
Le nom des modèles possède enfin comme suffixe leur numéro chronologique.