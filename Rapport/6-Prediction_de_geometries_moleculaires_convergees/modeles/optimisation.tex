
\label{delta_dist_quadri}

\par En plus du choix des données d'entrée, la performance des réseaux de neurones dépend de nombreux paramètres (\ref{apprentissage_automatique_parametres_nn}). Les résultats des modèles décrits dans ce chapitre étant peu probants (\ref{delta_dist_resultats}), j'ai effectué une recherche par quadrillage (\ref{apprentissage_automatique_quadri}) large des différents paramètres pour le modèle \emph{DELTA\_DIST\_+H\_05}, avec l'objectif de trouver un ensemble de paramètres menant à de meilleures performances. De même que pour les modèles décrits dans le chapitre précédent (\ref{dist_rel_quadri}), le temps d'exécution de l'entraînement d'un modèle limite grandement la possibilité d'entraîner des modèles avec des jeux de paramètres variés et un nombre élevé de validations croisées en un temps raisonnable. Il faut donc effectuer un compromis entre la quantité de modèles différents entraînés et le nombre d'entraînements de chacun de ces modèles. L'objectif ici est de trouver un jeu de paramètres menant à de bonnes performances, dans l'idée de le perfectionner et de le valider par la suite s'il existe. C'est pourquoi la priorité est donnée au nombre de jeux de paramètres différents plutôt qu'au nombre de validations de chacun de ces jeux.

\par Cette recherche par quadrillage est toutefois relativement large car elle est composée d'une grille (tableau \ref{t_grille_delta_dist}) décrivant les paramètres de 576 modèles différents avec une validation croisée à deux plis, soit un total de 1152 entraînements. La grille se veut également large car elle fait varier la plupart des paramètres avec des amplitudes élevées.

\begin{table}
	\centering
	
	\begin{tabular}{|l|l|}
		\hline
		\textbf{Paramètres} & \textbf{Valeurs} \\ \hline 
		Taux d'apprentissage (\textit{learning rate}) & 0.1, 0.0001, 0.00001 \\ \hline
		Epsilon (Adam) & 1000, 0.0001 \\ \hline
		Initialisation poids & 0.2, 0.002 \\ \hline
		Fonction d'activation couches cachées & elu, crelu \\ \hline
		Fonction d'activation couche de sortie & linéaire \\ \hline
		Dégradation des coefficients (\textit{weight decay}) & 0.1, 0.01, 0.001 \\ \hline
		Largeur & 500 \\ \hline
		Profondeur & 7, 3 \\ \hline
		Taille de lot (\textit{batch size}) & 500, 2000\\ \hline
		Époques d'entraînement & 3 \\ \hline
		
	\end{tabular}		
	
	\caption{Grille de recherche par quadrillage pour le modèle \emph{DELTA\_DIST\_+H\_05}}
	\label{t_grille_delta_dist}
\end{table}

\par À l'issue de la recherche, aucun ensemble de paramètres n'a mené à de meilleures performances que les modèles précédemment entraînés.
