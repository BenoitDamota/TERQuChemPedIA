\par Les modèles décrits dans ce chapitre sont tous des réseaux de neurones possédant des architectures simples. Ils sont composés d'une entrée et d'une sortie dont la taille dépend des données qu'ils doivent traiter (\ref{delta_dist_donnees}), et d'un certain nombre de couches internes de tailles fixes et entièrement connectées, c'est à dire que chaque neurone d'une couche est connecté à tous les neurones de la couche suivante.

\par Le nombre de couches et le nombre de neurones par couche varie en fonction des modèles. Les premiers modèles possédaient des couches internes plus larges que les entrées et sorties, ce qui pouvait potentiellement apporter un gain de performances mais qui augmentait de manière significative le temps d'entraînement. C'est pourquoi le dernier modèle est composé de couches internes de même taille que la couche d'entrée. Le détail est disponible dans la table des paramètres en annexe \ref{annexes_param_delta_dist}.