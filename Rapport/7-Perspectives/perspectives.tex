\section{Prédiction des géométries optimisées complètes}

\par Les modèles présentant les meilleurs résultats pour optimiser les géométries moléculaires résolvent en réalité des sous-problèmes de cette optimisation (REF DIST REL). Ils prédisent en effet la distance entre des couples d'atomes partageant des liaisons covalentes. Afin d'utiliser des modèles de ce type pour optimiser des molécules, il faudrait d'une part entraîner des modèles permettant de prédire la longueur de chaque type de liaison, et d'autre part mettre en place un système permettant d'utiliser ces prédictions locales pour optimiser la géométrie complète.\\


\par Pour prédire la géométrie complète à partir des modèles prédisant les longueurs des liaisons entre les différents couples d'atomes, nous pouvons imaginer un système dans lequel ces modèles formeraient différents modules permettant d'améliorer localement la géométrie. Nous pourrions alors les intégrer au sein d'un algorithme itératif qui améliorerait progressivement la géométrie, jusqu'à ce qu'un critère de convergence soit atteint. Cet algorithme constituerait une méthode comparable à l'optimisation quantique (REF OPTI GEOM), à la différence que les calculs coûteux seraient remplacés par les prédictions des modèles.\\

\par Certains modules pourraient être entraînés à prédire les longueurs de liaisons entre plusieurs couples d'atomes différents. Ils pourraient par exemple prédire les longueurs de liaisons entre des couples partageant un même atome, ou entre un atome et les atomes d'une même colonne du tableau périodique des éléments\footnote{https://fr.wikipedia.org/wiki/Tableau\_périodique\_des\_éléments}, ceux-ci partageant des propriétés similaires. Cela permettrait de réduire la quantité de modules différents nécessaire et donc de limiter la complexité globale du système d'optimisation géométrique.\\
Notons que tous les couples d'atomes du tableau périodique ne doivent pas être pris en compte, certains couples d'atomes ne pouvant pas partager de liaison.\\


\par Concernant la composition des modules, nous pouvons imaginer utiliser plusieurs types de modèles différents. En effet, si l'utilisation de réseaux de neurones artificiels s'avère très efficace pour prédire les longueurs de liaisons entre des couples d'atomes très représentés dans les données (REF GENERALISATION), il est probable que les résultats se dégradent considérablement lorsque l'on tentera de prédire des distances entre des couples d'atomes pour lesquels nous disposons de peu de données. Les réseaux de neurones ont en effet généralement besoin d'un grand nombre d'exemples d'apprentissage pour être efficaces, ce qui n'est pas le cas de tous les modèles prédictifs. Pour prédire les longueurs de liaisons des couples d'atomes peu représentés, nous pourrons notamment utiliser des modèles de type Kernel Ridge Regression, qui semblent également efficaces et qui s'entraînent sur des jeux de données plus réduits (REF KRR).\\


\section{Représentation des données moléculaires}
		
\paragraph{Graphes} Nous utilisons actuellement une représentation des molécules sous forme de tableau de caractéristiques décrivant indépendamment chacun des atomes (ENTREE DELTA DIST). Une représentation de ce type présente l'avantage d'être simple à mettre en place en tant qu'entrée d'un modèle prédictif, mais elle n'est pas très adaptée pour représenter des ensembles d'objets interagissant les uns avec les autres, comme les atomes d'une molécule. Une représentation sous forme de graphe serait en effet plus naturelle. Les nœuds permettraient de représenter les atomes, et les arêtes représenteraient les différentes interactions, dont font partie les liaisons covalentes.
\par Si l'on se place dans le référentiel d'une liaison covalente (REF ENTREE DIST REL), on peut également imaginer une représentation sous forme de graphe, dans laquelle les nœuds correspondraient aux différents atomes au voisinage, et les arêtes représenteraient de même les interactions entre les atomes de la liaison et les atomes du voisinage.

\par La représentation par graphe pose toutefois certains problèmes techniques lorsque l'on veut l'utiliser comme entrée des réseaux de neurones artificiels. Ceux-ci possèdent en effet des entrées de taille fixe, alors que les graphes ont une taille variable, qui dépend du nombre d'atomes et des différentes liaisons. Il existe toutefois des techniques documentées permettant d'utiliser des graphes avec des réseaux de neurones convolutifs\cite{mg}, et permettant de prédire diverses propriétés chimiques\cite{jctc_prediction}. 
\paragraph{Fingerprints} Une technique communément utilisée pour représenter des molécules en entrée des modèles d'apprentissage automatique est la génération de \emph{fingerprints} (empreintes) des molécules. Il s'agit de méthodes de hachage permettant de représenter la composition et la structure des molécules sous forme d'une chaîne. Certaines méthodes ont de plus la propriété de fournir des empreintes de taille fixe. On peut notamment utiliser le programme RDKit\cite{rdkit} pour générer ces chaînes, à partir de différentes représentations en entrée.
\par Les représentations par graphe et par empreinte ne sont pas nécessairement incompatibles, puisque des méthodes permettent de générer des empreintes à partir de graphes en utilisant des réseaux de neurones convolutifs\cite{graph_fingerprint}.\\

\par Ces différentes représentations présentent des perspectives intéressantes pour le projet QuChemPedia (REF QUCHEMPEDIA). Nous pourrions en effet nous en inspirer afin de concevoir des modèles plus complexes, qui pourraient potentiellement être plus efficaces pour prédire les longueurs de liaisons, notamment lorsque nous utiliserons des données non convergées en entrée des modèles (REF MOTIVATION DIST REL).





