\addcontentsline{toc}{chapter}{Conclusion}  

\par Le travail que j'ai effectué s'inscrit dans le cadre d'un projet de recherche ambitieux et très intéressant. Il constitue une étape préliminaire à la réalisation d'un système davantage abouti, pour lequel il ouvre des perspectives encourageantes.\\

\par Lors de ce projet, je me suis efforcé d'appliquer une méthodologie stricte, dans le but de fournir une contribution fiable et utilisable pour la poursuite du projet QuChemPedia. Cet exercice s'est avéré complexe mais très instructif. Les résultats issus de mon travail sont toutefois à confirmer. Il faudrait d'une part s'assurer qu'aucune erreur méthodologique n'a été commise, et d'autre part valider les différents modèles avec de multiples exécutions.\\

\par Si le domaine de l'apprentissage automatique m'intéresse particulièrement, je n'avais jamais eu l'occasion de travailler sur des réseaux de neurones artificiels auparavant. J'ai par conséquent acquis un certain nombre de connaissances, que j'espère maintenant approfondir et mettre à profit par la suite.

\par En outre, la dimension collaborative du travail sur un projet de cette ampleur constitue un élément stimulant, qui participe à le rendre plaisant et gratifiant.\\

\par Pour toutes ces raisons, je souhaite continuer de travailler sur ce projet si j'en ai l'occasion, et participer à la création des systèmes plus complets en constituant la suite logique.